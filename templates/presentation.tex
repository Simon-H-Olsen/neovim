% Introductory sample article: intrart.tex

\documentclass{beamer}
\usetheme{Berkeley}

\begin{document}
\title[Complete-simple distributive lattices]% 
{A construction of complete-simple\\  
       distributive lattices}
	   \author[]{Simon~H. Olsen}
\institute{Computer Science Department\\
         University of Winnebago\\
         Winnebago, MN 53714} 
\date{\today}
%\thanks{sdlfkj}

\begin{frame}
\titlepage
\end{frame}

\begin{frame}
	\frametitle{Outline}
	\tableofcontents[pausesections,pausesubsections]
\end{frame}

\begin{frame}
	\frametitle{jsldkf}
\section{Introduction}\label{S:intro} 
In this note, we prove the following result:

\begin{theorem} 
There exists an infinite complete distributive 
lattice~$K$ with only the two trivial complete 
congruence relations.
\end{theorem}
\end{frame}

\begin{frame}
	\frametitle{jsldkf}
	\section[Construction]{The \texorpdfstring{$\pi^{*}$} construction}\label{S:P*} 
The following construction is crucial in the proof
of our Theorem (see Figure):

\begin{definition}\label{D:P*} 
Let $D_{i}$, for $i \in I$, be complete distributive 
lattices satisfying condition~\textup{(J)}.  Their 
$\pi^{*}$ product is defined as follows:
\[
   \pi^{*} ( D_{i} \mid i \in I ) = 
   \pi ( D_{i}^{-} \mid i \in I ) + 1;
\]
that is, $\pi^{*} ( D_{i} \mid i \in I )$ is 
$\pi ( D_{i}^{-} \mid i \in I )$ with a new 
unit element. 
\end{definition}
\end{frame}

\begin{frame}
	\frametitle{jsldkf}
If $i \in I$ and $d \in D_{i}^{-}$, then
\[
  \langle \dots, 0, \dots, d, \dots, 0, \dots \rangle
\]
is the element of $\pi^{*} ( D_{i} \mid i \in I )$ whose 
$i$-th component is $d$ and all the other components 
are $0$.

See also Ernest~T. Moynahan~\cite{eM57a}.
\end{frame}

\begin{frame}
	\frametitle{jsldkf}
Next we verify the following result:

\begin{theorem}\label{T:P*} 
Let $D_{i}$, $i \in I$, be complete distributive 
lattices satisfying condition~\textup{(J)}.  
Let $\Theta$ be a complete congruence relation on 
$\pi^{*} ( D_{i} \mid i \in I )$. 
If there exist $i \in I$ and $d \in D_{i}$ with 
$d < 1_{i}$ such that, for all $d \leq c < 1_{i}$, 
\begin{equation*}\label{E:cong1} 
   \langle \dots, d, \dots, 0, \dots \rangle \equiv 
   \langle \dots, c, \dots, 0, \dots \rangle 
   \pod{\Theta}, 
\end{equation*}
then $\Theta = \iota$.
\end{theorem}
\end{frame}

\begin{frame}
	\frametitle{jsldkf}
\centering\includegraphics{products}
\end{frame}

\begin{frame}
	\frametitle{jsldkf}
Since 
\begin{equation*}\label{E:cong2}
\langle \dots, d, \dots, 0, \dots \rangle \equiv 
\langle \dots, c, \dots, 0, \dots \rangle 
\pod{\Theta}, 
\end{equation*}
and $\Theta$ is a complete congruence relation, 
it follows from condition~(J) that
\begin{equation*}\label{E:cong}
 \langle \dots, d, \dots, 0, \dots \rangle \equiv
 \bigvee ( \langle \dots, c, \dots, 0, \dots \rangle 
 \mid d \leq c < 1 ) \pod{\Theta}. 
\end{equation*}
\end{frame}

\begin{frame}
	\frametitle{jsldkf}
Let $j \in I$, $j \neq i$, and let $a \in D_{j}^{-}$. 
Meeting both sides of the congruence \eqref{E:cong2} 
with $\langle \dots, a, \dots, 0, \dots \rangle$, 
we obtain that
\begin{equation*}\label{E:comp}
   0 = \langle \dots, a, \dots, 0, \dots \rangle 
     \pod{\Theta}, 
\end{equation*}
Using the completeness of $\Theta$ and \eqref{E:comp}, 
we get:
\[
   0 \equiv \bigvee ( \langle \dots, a, \dots, 0, 
     \dots \rangle \mid a \in D_{j}^{-} ) = 1 
     \pod{\Theta}, 
\]
hence $\Theta = \iota$.
\end{frame}

\begin{frame}
	\frametitle{jsldkf}
\begin{thebibliography}{9}

\bibitem{sF90}
Soo-Key Foo, 
\emph{Lattice Constructions}, 
Ph.D. thesis, 
University of Winnebago, Winnebago, MN, December, 1990.

\bibitem{gM68}
George~A. Menuhin, 
\emph{Universal algebra}.
D.~Van Nostrand, Princeton, 1968.

\bibitem{eM57}
Ernest~T. Moynahan, 
\emph{On a problem of M. Stone},
Acta Math. Acad. Sci. Hungar. \textbf{8} (1957), 
455--460.

\bibitem{eM57a}
Ernest~T. Moynahan, 
\emph{Ideals and congruence relations in lattices.} II,
Magyar Tud. Akad. Mat. Fiz. Oszt. K\"{o}zl. \textbf{9} 
(1957), 417--434.

\end{thebibliography}
\end{frame}

\end{document}

